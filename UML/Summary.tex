\documentclass{article}

\begin{document}
\section*{Example: Queueing Model}
This document provides an illustrative example of a queuing system, including its structure, current system state, simulation variables, and statistical counters. It serves as a reference point for understanding the components and metrics involved in the simulation.

\section*{Advanced Time / Handle Event}
This document outlines the advanced time handling and event processing mechanisms within the queuing system simulation. It covers the incremental updating of total wait time, as well as branching based on event types. Understanding this step is crucial for managing the temporal dynamics and event-driven nature of the simulation.

\section*{Arrival Event}
Here, the arrival event mechanism is detailed, focusing on the incremental update of the number of customers in the system and branching based on specific conditions. This step is essential for modeling the arrival process accurately and dynamically adjusting the system state in response to incoming customers.

\section*{Departure Event}
The departure event document describes the process of handling customer departures from the queuing system. It includes details on updating system states, such as customer departure counts, and managing event branching based on departure conditions. Mastering this step ensures proper management of customer flow and resource utilization within the simulation.

\end{document}

